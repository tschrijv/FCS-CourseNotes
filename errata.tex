\documentclass[11pt]{fund}
\usepackage[dutch]{babel}
% \usepackage{a4wide}
\usepackage{epsfig}
\usepackage{wrapfig}
\usepackage{subfigure}
\usepackage{graphics}
\usepackage{subfigure}
\renewcommand{\thepage}{\hspace*{-3cm}\arabic{page}}
\setlength{\textwidth}{18cm}
\setlength{\oddsidemargin}{-1.5cm}


%fancy heading
\usepackage{fancyhdr}
\pagestyle{fancy}


%  Een paar definities
\renewcommand{\thesection}{\arabic{section}}
\newcommand{\ds}{\displaystyle}
\newcommand{\bin}[2]{{{\ds #1} \choose {\ds #2}}}
\newcommand{\prend}{\hfill \rule{2.3mm}{2.3mm} \hspace*{3mm} $ $}
\newcommand{\rpijl}{$\rightarrow~$}
\newcommand{\eps}{$\epsilon~$}
\newcommand{\project}[1]{}

\newcommand{\section}[1]{\setcounter{mijnstelling}{1} \section{#1}}

% veranderd door Bart Demoen: eerst degene die een nummer krijgen
% \newtheorem{stelling}[subsection]{Stelling}
% \newtheorem{definitie}[subsection]{Definitie}
% \newtheorem{informeledefinitie}[subsection]{Informele definitie}
% \newtheorem{lemma}[subsection]{Lemma}
% \newtheorem{algo}[subsection]{Algoritme}
% \newtheorem{eig}[subsection]{Eigenschap}
% \newtheorem{gevolg}[subsection]{Gevolg}

\lhead[\fancyplain{}{\rm\thepage}]{\fancyplain{}{\rightmark}}
\chead[]{}
\rhead[\fancyplain{}{\leftmark}]{\fancyplain{}{\rm\thepage}}
\lfoot[]{}
\cfoot[]{\fancyplain{\rm\thepage}{}}
\rfoot[]{}

\newcounter{mijnstelling}

\newenvironment{definitie}{{\bf Definitie \thesection.\themijnstelling~~} \addtocounter{mijnstelling}{1}}{}

\newenvironment{stelling}{{\bf Stelling \thesection.\themijnstelling~~} \addtocounter{mijnstelling}{1}}{}

\newenvironment{informeledefinitie}{{\bf Informele definitie \thesection.\themijnstelling~~} \addtocounter{mijnstelling}{1}}{}

\newenvironment{algo}{{\bf Algoritme \thesection.\themijnstelling~~} \addtocounter{mijnstelling}{1}}{}

\newenvironment{lemma}{{\bf Lemma \thesection.\themijnstelling~~} \addtocounter{mijnstelling}{1}}{}

\newenvironment{gevolg}{{\bf Gevolg \thesection.\themijnstelling~~} \addtocounter{mijnstelling}{1}}{}



% nu degene zonder nummer
\newtheorem{vb}{Voorbeeld}
\renewcommand{\thevb}{}

\newtheorem{opm}{Opmerking}
\renewcommand{\theopm}{}

\newtheorem{oef}{Oefening}
\renewcommand{\theoef}{}

\newtheorem{leeg}{\mbox{\hspace*{-0.45cm}}}
\renewcommand{\theleeg}{}

\setlength{\parindent}{0cm}

\begin{document}

%% TITELBLAD
\renewcommand{\thepage}{}
% \pagestyle{plain}


\renewcommand{\thepage}{\arabic{page}}
De bladzijde verwijst naar de originele versie van de cursus (3
augustus) die door Wina in omloop is gebracht.


\begin{tabular}{|r|r|p{6cm}|p{6cm}|p{2cm}|}
\hline
blz & datum    & fout & goed  & auteur \\ \hline
\hline
7 & 4-10-07 & N~is~een~priemgetal  & n~is~een~priemgetal  & J. Wittocx    \\
\hline
9 & 4-10-07 & beschrijving van $(ab)*$ is mis    & {\em elke a wordt door een b gevolgd en er zijn evenveel a's als b's}                     & J. Wittocx    \\
\hline
16 & 4-10-07 & transitietabel vermeldt c niet  &  is toegevoegd                     & J. Wittocx    \\
\hline
17 & 4-10-07 & aan een eindtoestand &  aan \'e\'en eindtoestand, waaruit bovendien geen pijlen vertrekken                    & J. Wittocx    \\
\hline
23 & 4-10-07 &  In stap 2: $E_2*$ &  $E_2^*$                       & J. Wittocx    \\
\hline
25 & 4-10-07 & In punt 1: AXB   & AX$^n$B ($n>0$)   & J. Wittocx    \\
\hline
27 & 4-10-07 & die met een $\epsilon$-boog bereikbaar zijn & die met \'{e}\'{e}n of meer $\epsilon$-bogen bereikbaar zijn & J. Wittocx    \\
\hline
27 & 4-10-07 & $\delta_d({\cal Q},a) = \delta_n(eb({\cal Q}),a)$  & $\delta_d({\cal Q},a) = eb(\delta_n({\cal Q},a))$  & B. Demoen    \\
\hline
28 & 4-10-07 & de fout op pagina 27 sijpelde door in tabel 2.2 en fig 2.11 (p.29) & de hele tabel is aangepast en ook figuur 2.11 & B. Demoen    \\
\hline
34 & 4-10-07 &  sj                    &   $s_j$                   & J. Wittocx    \\
\hline
40 & 4-10-07 &  de tweede $\Longleftrightarrow$ in punt 2 was mis  & een aantal pijlen zijn aangepast en de definitie van $\sim_{sup}$ is ingevoerd   & J. Wittocx    \\
\hline
50 & 11-10-07 & de drie automaten aanvaarden te veel strings (bijvoorbeeld 110)   & in elke automaat is \'{e}\'{e}n eindtoestand gewoon gemaakt & R. Strackx    \\ \hline
66 & 17-10-07  & $\Gamma_\epsilon^*$ & $\Gamma^*$ & B. Demoen \\ \hline
67 & 17-10-07 &     {\em maar de klasse van talen ...} &     {\em maar {\bf niet} de klasse van talen ...} & B. Demoen \\ \hline


80 & 17-10-07 & de grammatica voor $a_nb_nc_n$ voldoet niet aan de originele definitie van context-sensistief   & wel aan een equivalente definitie - dat is toegevoegd & B. Demoen    \\
\hline

43 &
23-10-07 &
$(q_s,q_1,a_2,...,q_f)$ &
$(q_s,q_1,q_2,...,q_f)$ &
J. Wittocx \\ \hline

63 &
23-10-07 &
stap 5 van de transformatie naar Chomsky NF is onvolledig gespecifieerd &
... door $A \rightarrow X_1Y_1$, $Y_1 \rightarrow X_2Y_2$, ...,
$Y_{n-2} \rightarrow X_{n-1}X_n$ &
J. Wittocx \\ \hline

66 &
23-10-07 &
{\em en zitten we in een eindtoestand} &
{\em en nog \'{e}\'{e}n toepassing brengt ons in de eindtoestand} &
J. Wittocx \\ \hline

66 &
23-10-07 &
$stack_i \in \Gamma$ &
$stack_k \in \Gamma$ &
J. Wittocx \\ \hline

74 &
23-10-07 &
uvvyyz &
uvvxyyz &
J. Wittockx \\ \hline \hline
\end{tabular}

\begin{tabular}{|r|r|p{6cm}|p{6cm}|p{2cm}|}
\hline
blz & datum    & fout & goed  & auteur \\ \hline
\hline

15 &
23-11-07 &
def 6.4, puntje twee - argument van delta vergeten en index verkeerd &
zie nieuwe versie &
Stijn Vermeeren \\ \hline

27 &
23-11-07 &
de tekst na de definitie van $\delta_d$ was nog niet aangepast aan de
verbetering van de definitie zelf &
zie nieuwe versie &
anoniem in de les \\ \hline

34 &
23-11-07 &
in de regel die met {\em Maar:} begint ontbreekt $p_s$ in 2de en 3de term &
zie nieuwe versie &
Ruth Nysen \\ \hline

46 &
23-11-07 &
$Q_1 \setminus F_2$ &
$Q_1 \setminus F_1$ &
Ruth Nysen \\ \hline

51 &
23-11-07 &
bij elke overgang \'{e}\'{e}n positie naar opschuift &
bij elke overgang \'{e}\'{e}n positie naar rechts opschuift &
Ruth Nysen \\ \hline

66 &
23-11-07 &
def 20.2 - indexen en x,y mis&
zie nieuwe versie &
Stijn Vermeeren \\ \hline

119 &
27-11-07 &
Vermits $A_{TM}$ niet herkenbaar is &
Vermits $\overline{A_{TM}}$ niet herkenbaar is &
Johan Wittocx \\ \hline

38 &
27-11-07 &
De verzameling $\{reach(q) | q \in Q\}$ is een partitie van $\Sigma^*$ &
toevoegen: als elke toestand bereikbaar is &
Stijn Vermeeren \\ \hline

96 &
27-11-07 &
in de definitie van $A_{TM}$ staat $s \in L_{TM}$ &
moet $s \in L_M$ zijn &
Stijn Vermeeren \\ \hline

70 &
3-12-07 &
in de derde overgang E+E te kiezen i.p.v. E$*$E &
in de derde overgang E*E te kiezen i.p.v. E$+$E &
Stijn Vermeeren \\ \hline


40 &
10-12-07 &
de definitie 13.5 vergat dat ook transitieve sluiting moet genomen worden &
de definitie en setlling 13.6 zijn herschreven &
Bart Demoen \\ \hline


\end{tabular}

\newpage
\begin{tabular}{|r|r|p{6cm}|p{6cm}|p{2cm}|}
\hline
blz & datum    & fout & goed  & auteur \\ \hline

\hline
122  &
12-12-2007  &
${p_n}^i$  &
${p_i}^n$  (2 maal) &
Stijn Vermeeren  \\ \hline

\hline
123  &
12-12-2007  &
$Pr[{p_1}^1,Cn[s,{p_3}^3]]$  &
$Pr[{p_1}^1,Cn[suc,{p_3}^3]]$  &
Stijn Vermeeren  \\ \hline

\hline
123  &
12-12-2007  &
$Pr[x,Cn[som,{p_1}^3, {p_3}^3]]$  &
$Pr[nul,Cn[som,{p_1}^3, {p_3}^3]]$  &
Stijn Vermeeren  \\ \hline

\hline
122-123 &
12-12-2007 &
de definitie van primitieve recursie is onvolledig &
het geval k=0 is toegevoegd (zie verbeterde tekst) &
Stijn Vermeeren  \\ \hline


\hline
90 &
19-12-2007 &
$\alpha q b \beta$ \rpijl $\alpha a c p \beta$ indien $\delta(q,b) = (p,c,R)$ &
$\alpha q b \beta$ \rpijl $\alpha c p \beta$ indien $\delta(q,b) = (p,c,R)$ &
Gutierrez Gonzalez, Andres Humberto  \\ \hline

\hline
39 &
19-12-2007 &
Stelling 13.3 was vaag over waarvoor inductie moest gebruikt worden &
nieuwe versie is explicieter &
Bart Demoen  \\ \hline

\hline
155 &
19-12-2007 &
i.v.m. de typering van Y stond er wat verwarrende onzin &
nieuwe versie is beter &
Sarah Wauters  \\ \hline


\end{tabular}

\newpage
\begin{tabular}{|r|r|p{6cm}|p{6cm}|p{2cm}|}
\hline
blz & datum    & fout & goed  & auteur \\ \hline

114 &
2-1-2008 &
\$Aab\$ helemaal onderaan blad & 
\$Xab\$ &
Ruth Nysen  \\ \hline

119 &
2-1-2008 &
als volgt: met $<M,w>$ laten we & 
als volgt: met $<M,s>$ laten we & 
Ruth Nysen  \\ \hline

120 &
2-1-2008 &
Omgekeerd: als $L_M \neq \emptyset$, & 
Omgekeerd: als $L_M = \emptyset$, &
Ruth Nysen  \\ \hline

62 &
2-1-2008 &
$A \rightarrow \gamma$ waar $\gamma$ meer dan twee symbolen beva & 
$A \rightarrow \gamma$ waar $\gamma$ minstens twee symbolen beva &
Nick Vannieuwenhoven  \\ \hline

117 &
2-1-2008 &
$f(\overline{L_2}) \subseteq \overline{L_2}$ & 
$f(\overline{L_1}) \subseteq \overline{L_2}$ &
Nick Vannieuwenhoven  \\ \hline

71 &
2-1-2008 &
regels van de vorm $A_{p,q} \rightarrow A_{p,r}A_{r,p}$ & 
regels van de vorm $A_{p,q} \rightarrow A_{p,r}A_{r,q}$ &
Raoul Strackx, Nick Vannieuwenhoven  \\ \hline


\end{tabular}

\newpage
\begin{tabular}{|r|r|p{6cm}|p{6cm}|p{2cm}|}
\hline
blz & datum    & fout & goed  & auteur \\ \hline

85 &
3-1-2008 &
$\Sigma$ in de tabellen moet zijn: &
$\Gamma$ &
Sarah Wauters  \\ \hline

66 &
27-1-2008 &
$(q_{i+1},y) \in \delta(q_i,w_i,x)$&
$(q_{i+1},y) \in \delta(q_i,w_{i+1},x)$&
Nick Vannieuwenhoven \\ \hline

\end{tabular}

\newpage
\begin{tabular}{|r|r|p{6cm}|p{6cm}|p{2cm}|}
\hline
blz & datum    & fout & goed  & auteur \\ \hline

47 &
10-11-2008 &
$(+|-|\epsilon)PosCijfer~Cijfer^*$ &
$(+|-|\epsilon)PosCijfer~Cijfer^*~|~0$ &
Michaël Vanderheeren \\ \hline

55 &
10-11-2008 &
Q \rpijl cPd &
Q \rpijl cQd &
Michaël Vanderheeren \\ \hline

117 &
23-11-2008 &
in definitie 16.2 $f(\overline{L_2}) \subseteq \overline{L_2}$ &
$f(\overline{L_1}) \subseteq \overline{L_2}$ &
Michaël Vanderheeren \\ \hline

\end{tabular}




Veel taalfouten/tikfouten zijn hierboven niet opgenomen (en soms nog
niet verbeterd in de on-line versie) omdat ze niet zo essentieel zijn
voor het verstaan van de cursus. Toch dank voor het melden ervan aan
Ruth Nysen, Dieter Van Melkebeek, Stijn Vermeeren, Thomas De Craemer,
Pieter Agten, Willem Penninckx, Lore Kesteloot en vooral Johan Wittocx.

\end{document}


princiepes/principes ...
