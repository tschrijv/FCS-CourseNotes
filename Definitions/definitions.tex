\chapter*{Some definitions}

There are some mathematical definitions that you might be unfamiliar with, that are required for understanding the course material. These are covered here.

\subsection*{Partial and total function}

A function $ f: X \mapsto Y $ is a partial function if it is only defined for the domain $X' \subseteq X$. If $X' = X$, $f$ is a total function.

\subsection*{Multiset}

A multiset is similar to a set, but an element can occur more than once, for example: $\{1, 1, 1, 2, 2, 3, 4, 5, 5\}$.

\subsection*{Powerset}

The powerset of a set $S$ is the set of all subsets of $S$.

For example:

\begin{align*}
S &= \{1, 2, 3\} \\
{\cal P}(S) &= \{\emptyset, \{1\}, \{2\}, \{3\}, \{1, 2\}, \{1, 3\}, \{2, 3\}, \{1, 2, 3\}\}
\end{align*}

\subsection*{Equivalence relation}

A binary relation R on a set $S$ is an equivalence relation if and only if it is
\begin{enumerate}
\item reflexive $ a = a $
\item symmetric $ a = b \Leftrightarrow b = a $
\item transitive $ a = b \land b = c \Rightarrow a = c $
\end{enumerate}

\subsection*{Order relations}

A binary relation $R$ on a set $S$ is a \em{partial} order if and only if it is
\begin{enumerate}
\item reflexive $ a \leq a $
\item antisymmetric $ a \leq b \land b \leq a \Rightarrow a = b $
\item transitive $ a \leq b \land b \leq c \Rightarrow a \leq c $
\end{enumerate}

A binary relation $R$ on a set $S$ is a \em{total} order if and only if it is
\begin{enumerate}
\item a partial order
\item for any pair of elements $a$ and $b$ of $S$: $ (a, b) \in R \lor (b, a) \in R $
\end{enumerate}

\subsection*{Transitive closure}

The transitive closure of a binary relation $R$ on a set $S$ is the smallest relation on $S$ that contains $R$ and is transitive.

For example:

\begin{align*}
S &= \{1, 2, 3\} \\
R &= \{(1, 2), (2, 3)\} \\
T &= \{(1, 2), (2, 3), (1, 3)\}
\end{align*}
