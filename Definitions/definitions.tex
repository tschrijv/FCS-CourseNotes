\chapter*{Prerequisites}

There are several basic mathematical notions used
throughout this course that you should be familiar with.
These are briefly summarized here as a refresher.

%-------------------------------------------------------------------------------
\section*{Sets, Relations and Functions}

%- - - - - - - - - - - - - - - - - - - - - - - - - - - - - - - - - - - - - - - - 
\subsection*{Cartesian Products}
The Cartesian product of two sets $A$ and $B$, which is denoted $A \times B$, is
the set of all tuples that can be formed by pairing an element of $A$ with one
of $B$:
\[ A \times B = \{ (x,y) \mid x \in A, y \in B \} \]

%- - - - - - - - - - - - - - - - - - - - - - - - - - - - - - - - - - - - - - - - 
\subsection*{Powersets}
The powerset of a set $S$, denoted $\mathcal{P}(S)$ or $2^S$,  is the set of
all subsets of $S$.

For example:
\begin{equation*}\begin{aligned}
	S &= \{1, 2, 3\} \\
	{\cal P}(S) &= \{\emptyset, \{1\}, \{2\}, \{3\}, \{1, 2\}, \{1, 3\}, \{2, 3\}, \{1, 2, 3\}\}
\end{aligned}\end{equation*}

%- - - - - - - - - - - - - - - - - - - - - - - - - - - - - - - - - - - - - - - - 
\subsection*{Multisets}

A multiset is similar to a set, but an element can occur more than once, for
example: $\{1, 1, 1, 2, 2, 3, 4, 5, 5\}$. We call the number of times an 
element appears in a multiset that element's \emph{multiplicity}.

%- - - - - - - - - - - - - - - - - - - - - - - - - - - - - - - - - - - - - - - - 
\subsection*{Predicates and Relations}

Informally, a predicate $P$ over a set $A$ denotes a property that
elements of set $A$ may have. We write $P(x)$, with $x \in A$ to denote that
element $x$ has the property. Formally, a predicate $P$ over a set
$A$ is just a subset of $A$; it is the subset of all elements of $A$ that have
the property. Hence, $P(x)$ is just an alternative notation for $x \in P$.

%- - - - - - - - - - - - - - - - - - - - - - - - - - - - - - - - - - - - - - -
\subsection*{Binary Relations}

A binary relation $\mathcal{R}$ over sets $A$ and $B$ is a predicate over $A
\times B$. We say that $x \in A$ is related to $y \in B$ when
$\mathcal{R}(x,y)$ holds. Sometimes, we also write $x\mathcal{R}y$ instead of
$\mathcal{R}(x,y)$.

%- - - - - - - - - - - - - - - - - - - - - - - - - - - - - - - - - - - - - - -
\subsection*{Equivalence Relations}

A binary relation $\mathcal{R}$ on $S \times S$ is an \emph{equivalence} relation iff it satisfies three properties:
\begin{equation*}\begin{aligned}
	\text{reflexivity}  && &\forall x \in S: x\mathcal{R}x \\
	\text{symmetry}     && &\forall x, y \in S: x\mathcal{R}y \;\;\Leftrightarrow\;\; y\mathcal{R}x \\
	\text{transitivity} && &\forall x, y, z \in S: x\mathcal{R}y \;\wedge\; y\mathcal{R}z \;\;\Rightarrow\;\; x\mathcal{R}z
\end{aligned}\end{equation*}
An obvious example of an equivalence relation is equality ($=$) over numbers. Another equivalence is parallelism ($\,||\,$) over the set of straight lines.

An \emph{equivalence class} of $S$ is a subset $E\subseteq S$ so that the equivalence holds between all elements of $E$, and it doesn't between elements of $E$ and those in $S\setminus E$.

%- - - - - - - - - - - - - - - - - - - - - - - - - - - - - - - - - - - - - - -
\subsection*{Order Relations}

A binary relation $\mathcal{R}$ on $S \times S$ is a \emph{partial order} on $S$ iff it
has the following three properties:
\begin{center}
\begin{tabular}{r@{\hspace{1cm}}l}
reflexivity  & $\forall x \in S: x\mathcal{R}x$ \\
antisymmetry     & $\forall x, y \in S: x\mathcal{R}y \;\wedge\; y\mathcal{R}x \;\;\Rightarrow\;\; x = y$ \\
transitivity & $\forall x, y, z \in S: x\mathcal{R}y \;\wedge\; y\mathcal{R}z \;\;\Rightarrow\;\; x\mathcal{R}z$ 
\end{tabular}
\end{center}
A good example of a partial order is the non-strict subset relation ($\subseteq$) over sets.

A partial order $\mathcal{R}$ on $S$ is a \emph{total} order if and only if also satisfies a fourth
property:
\begin{center}
\begin{tabular}{r@{\hspace{1cm}}l}
connexity& $\forall x, y \in S: x\mathcal{R}y \vee y\mathcal{R}x$ 
\end{tabular}
\end{center}
That is, all elements can be compared by the order. Observe that connexity generalizes reflexivity. A typical example is less-than-or-equal-to ($\leq$) over the real numbers.

%- - - - - - - - - - - - - - - - - - - - - - - - - - - - - - - - - - - - - - -
\subsection*{Preorder Relations}
A \emph{preorder} is a binary relation $\mathcal{R}$ on $S \times S$ that has reflexivity and transitivity, but isn't consistently symmetric or anti-symmetric. Sometimes $x\mathcal{R}y$ and $y\mathcal{R}x$, sometimes only $x\mathcal{R}y$, and sometimes neither.

After grouping together the subsets for which symmetry does hold, i.e.\ the subset of $\mathcal{R}$ that is an equivalence relation, the preorder $\mathcal{R}$ is a partial order between the elements of the equivalence classes. This kind of relation appears often in computer science: in this course, we'll encounter the example of polynomial reducibility ($\leq_p$) and its symmetric subset, polynomial equivalence ($\sim_p$).

%- - - - - - - - - - - - - - - - - - - - - - - - - - - - - - - - - - - - - - -
\subsection*{Transitive Closure}

The transitive closure of a binary relation $\mathcal{R}$ on $S \times S$,
denoted $\mathcal{R}^+$, is
the smallest relation on $S \times S$ that contains $\mathcal{R}$ and that 
has the transitivity property. For example:
\begin{equation*}
\begin{aligned}
S &= \{1, 2, 3\} \\
\mathcal{R} &= \{(1, 2), (2, 3)\} \\
\mathcal{R}^+ &= \{(1, 2), (2, 3), (1, 3)\}
\end{aligned}
\end{equation*}

%-------------------------------------------------------------------------------
\subsection*{Partial and Total Functions}

A function $f: A \to B : x \mapsto f(x)$ is a binary relation on $A \times B$ where
we write $f(x) = y$ rather than $f(x,y)$ to denote that $x \in A$ and $y \in B$
are related. We tend to think of $x$ as the input and $y$ as the output. Moreover, a function satisifies the \emph{functional dependency}
property:
\[ \forall x \in A, \forall y_1, y_2 \in B: f(x) = y_1 \wedge f(x) = y_2 \;\;\Rightarrow\;\; y_1 = y_2 \]
We say that a function $f$ is \emph{total} iff:\footnote{The word \emph{iff} is short for ``if and only if''. It represents logical equivalence ($\Leftrightarrow$).}
\[ \forall x \in A, \exists y \in B: f(x) = y \]
If a function is not total, it is called \emph{partial}. Usually, when we do
not specify that a function is partial, we implicitly assume that it is total.


%-------------------------------------------------------------------------------
\subsection*{Injection, Surjection and Bijection}
A function maps the same input to one or no output. A binary relation $\mathcal{R}$ on $A\times B$ that maps to the same output once or never is called \emph{injective}. Formally:
\begin{equation*}
	\forall x_1,x_2\in A, \forall y\in B : \mathcal{R}(x_1,y)  \wedge \mathcal{R}(x_2,y) \;\;\Rightarrow\;\; x_1 = x_2
\end{equation*}
A binary relation that maps at least one input to each output is called \emph{surjective}. Formally:
\begin{equation*}
	\forall y \in B, \exists x\in A : \mathcal{R}(x,y)
\end{equation*}
A \emph{bijection} is a function that is both injective and surjective. It matches all outputs to exactly one input, and if the function is total (which is usually the case in computer science), we can uniquely translate back and forth between all elements of $A$ and $B$.

%-------------------------------------------------------------------------------
\section*{Proof Techniques}

%- - - - - - - - - - - - - - - - - - - - - - - - - - - - - - - - - - - - - - -
\subsection*{Pigeonhole Principle}
The \emph{pigeonhole principle} states that if there are $n$ pigeonholes and more than $n$ pigeons, at least one hole must contain two pigeons. This obvious fact can be very powerful already: for example, by applying the pigeon and hole metaphor correctly to graphs, it's easy to prove that simple graphs have at least two vertices with the same degree. (You should come back to try proving this once you've learnt what all those terms mean!)

%- - - - - - - - - - - - - - - - - - - - - - - - - - - - - - - - - - - - - - -
\subsection*{Proof by contradiction and contrapositive}
To prove a conditional statement $p \Rightarrow q$ for two other statements $p$ and $q$ (e.g.\ $p =$ ``$D$ is a dog'' and $q =$ ``$D$ has four legs''), we often use proofs that aren't direct:
\begin{itemize}
\item \emph{By contrapositive}: $p \Rightarrow q$ is equivalent to $\neg q \Rightarrow \neg p$, so we can prove that instead (check that this works for the example above).
\item \emph{By contradiction}: assume that $q$ is false even while $p$ is true (violating the implication). By proving that this situation allows deriving a falsehood (a logical contradiction like $0 = 1$), this must mean $\neg (p \wedge \neg q)$, equivalent to $\neg p \vee q$ and thus $p \Rightarrow q$.
\end{itemize}

%- - - - - - - - - - - - - - - - - - - - - - - - - - - - - - - - - - - - - - -
\subsection*{Weak Induction}

Weak induction is a proof technique that establishes $\forall n \in \mathbb{N}:
P(n)$, with $P$ some predicate on $\mathbb{N}$.
It requires proving two simpler statements:
\begin{itemize}
	\item[(a)] $P(0)$, and
	\item[(b)] $\forall n \in \mathbb{N}: P(n) \Rightarrow P(n+1)$.
\end{itemize}

%- - - - - - - - - - - - - - - - - - - - - - - - - - - - - - - - - - - - - - -
\subsection*{Strong Induction}
Strong induction is derived from weak induction. It requires proving two simpler statements, the second of which differs from weak induction:
\begin{itemize}
	\item[(a)] $P(0)$, and
	\item[(b')] $\forall n \in \mathbb{N}: (\forall m \leq n: P(m)) \Rightarrow P(n+1)$.
\end{itemize}
