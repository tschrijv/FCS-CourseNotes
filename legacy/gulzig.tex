

\section{Gulzige algoritmen en matroids}\label{matroids}

De algoritmen van Prim en Kruskal zijn gulzig: je sorteert \'{e}\'{e}n
keer de bogen - de mogelijke keuzes om een MOB op te bouwen - en dan
kies je telkens de kleinste om toe te voegen of je verwerpt die voor
altijd. Die keuze is onderworpen aan een beperking: in het ene geval
is dat {\em verbindt de huidige MOB met een knoop erbuiten}, in het
andere geval {\em veroorzaakt geen kring indien toegevoegd}. Het
bewijs dat dit werkt is ad hoc, in de zin dat het niet past in een
meer algemeen kader. Dat kader bieden we hier nu aan, en we laten ook
zien hoe Kruskal past in dit kader. Eerst wat definities en voorbeelden
die eraan voldoen.

\grijs{\begin{definitie} Deelverzamelingsysteem\\
{\rm
Voor een verzameling $S$ en een verzameling $D$ van deelverzamelingen
van $S$ \\ (dus $D \subseteq {\cal P}(S)$), is $(S,D)$ een
deelverzamelingsysteem alss
\begin{itemize}
\item[]
$\forall X, Y: (X \in D~ \& ~ Y\subset X) \rightarrow Y \in D$
\end{itemize}
 }
We zeggen ``D is gesloten onder inclusie''.

Als $D$ leeg is, dan noemen we het deelverzamelingsysteem triviaal.

Men noemt de elementen van $D$ dikwijls {\em onafhankelijk}.
\end{definitie}}

Hier zijn wat voorbeelden van deelverzamelingsystemen: $S$ is steeds
een willekeurige verzameling

\begin{itemize}
\item[{\bf Alles}:]
$(S,{\cal P}(S))$

\item[{\bf Zonder}:]
$(S,Zonder(s,S))$ waarbij $s \in S$ en $Zonder(s,S) = \{X \subset S| s \notin X\}$

\item[{\bf Onafh:}]
$(V,Onafh(V))$ waarbij $V$ een vectorruimte is en $Onafh(V)$ de
verzameling van alle verzamelingen van onafhankelijke vectoren in $V$

\item[{\bf GeenKring:}]
$(E,GeenKring(G))$ waarbij $E$ de boogverzameling is van een graaf
$G(V,E)$ en $GeenKring(G)$ de verzameling van deelverzamelingen van
$E$ die geen kring bevatten

\item[{\bf GeenBoog:}]
$(V,GeenBoog(G))$ waarbij $V$ de knoopverzameling is van een graaf
$G(V,E)$ en $GeenBoog(G)$ de verzameling van deelverzamelingen van
$V$ waartussen geen boog bestaat in $G$

\end{itemize}

Nu nog wat voorbeelden van structuren die geen deelverzamelingsysteem zijn:

\begin{itemize}
\item[{\bf Met}:]
$(S,Met(s,S))$ waarbij $s \in S$ en $Met(s,S) = \{X \subset S| s \in X\}$

(merk op: als $(S,D)$ een niet-triviaal deelverzamelingsysteem is, dan $\phi \in D$)

\item[{\bf Basis}:]
$(V,Basis(V))$ waarbij $V$ een vectorruimte is en $Basis(V)$ de
verzameling van alle basissen van $V$

\item[{\bf Boom:}]
$E,Boom(G))$ waarbij $E$ de boogverzameling is van een graaf $G(V,E)$
en $Boom(G)$ de verzameling van deelverzamelingen van $E$ die een boom
zijn

\end{itemize}

Als voor een deelverzamelingsysteem $(S,D)$ is aan de elementen van
$S$ een gewicht wordt gegeven, dan krijgen we in deze context een
natuurlijk optimalisatieprobleem 

\begin{itemize}
\item[]
zoek een element $O$ van $D$ met het grootste gewicht\footnote{het
gewicht van een verzameling is de som van de gewichten van zijn
elementen}
\end{itemize}
Zulk een $O$ noemen we een {\em maximum} element van $D$.

Dat optimalisatieprobleem kan je proberen op te lossen door een
generisch gulzig algoritme:

\begin{itemize}
\item orden de elementen van $S$ van groot naar klein, tot een rij $R$
die ge\"indexeerd is van 1 tot $|S|$

\item $O := \phi$;

\item $\underline{for}~ i = 1 .. |S|$ 
\begin{itemize}
\item[] ~~~$\underline{if}~ O \cup \{R[i]\} \in D$
\item[] ~~~~~~~~~$\underline{then}~ O := O \cup \{R[i]\}$
\end{itemize}
\end{itemize}

Dit algoritme geeft zeker een {\em maximaal} element in $D$ (je kan
geen element van $S$ meer toevoegen, of je bekomt iets dat buiten
$D$ ligt - pas op, om dat te bewijzen heb je de definitie van
deelverzamelingsysteem echt wel nodig !), maar misschien geeft het
geen {\em maximum} element, en dan lost het het optimalisatieprobleem
niet op.

\paragraph{Zelf doen}

\begin{itemize}
\item[] ga na of voor de 5 voorbeelden van deelverzamelingsystemen
hiervoor, bovenstaand algoritme ook een maximum element berekent - je
moet zelf de gewichten defini\"eren
\end{itemize}

\newpage

E\'{e}n bijkomende eigenschap van
een deelverzamelingsysteem garandeert dat het gulzige algoritme
hierboven, ook een maximum element berekent, gelijk wat de gewichten
zijn.

\grijs{\begin{definitie} Matroid\\
{\rm
Een matroid is een deelverzamelingsysteem $(S,D)$ met de bijkomende
eigenschap dat 
$\forall X, Y: (X \in D, Y \in D, |X| < |Y|) \rightarrow
\exists y \in Y \setminus X: X \cup \{y\} \in D$
}
\end{definitie}}


In woorden: als twee elementen van $D$ van verschillende grootte zijn,
dan zit er in de grootste een element dat je kan toevoegen aan de
kleinste en dan bekom je nog altijd een element van $D$. Pas op, {\em
grootte} betekent {\em aantal elementen}, niet {\em gewicht}: gewichten spelen (nog) geen rol.


Welke van de deelverzamelingsystemen hierboven is een matroid ?

\begin{lemma}
In matroid $(S,D)$ zijn alle maximale elementen (van $D$) even groot
\end{lemma}
\underline{Proof}
Stel dat $A$ en $B$ maximale elementen zijn van $(S,D)$, en dat $|A| <
|B|$, dan $\exists b \in B$ zodat $A \cup \{b\} \in D$ wat de
maximaliteit van $A$ tegenspreekt. Dus $|A| = |B|$.
\prend

Er is ook een soort omgekeerde van vorig lemma: voor een $(S,D)$ geldt

als $\forall A \subset D$ $(S,D)$ alle maximale subsets van $A$ even
groot zijn, dan is $(S,D)$ een matroid.

Het lemma bewees het omgekeerde enkel voor $A = D$.


\grijs{\begin{stelling}
Voor een deelverzamelingsysteem $(S,D)$ geldt:
het generisch gulzig algoritme lost het optimalisatieprobleem op voor
$(S,D)$ alss $(S,D)$ een matroid is.
\end{stelling}}
\underline{Proof}
TODO
\prend

Deze stelling verbindt een eigenschap van deelverzamelingsystemen met
een eigenschap van een algoritme: de eigenschap van
deelverzamelingsystemen refereert niet naar het algoritme, en zelfs
niet naar de gewichten, die toch essentieel zijn voor het algoritme en
het optimalisatieprobleem. Het algoritme maakt dan weer geen expliciet
gebruik van de specifieke eigenschap van de matroid. Sterk, nietwaar !

todo
\begin{itemize}
\item 
kruskal
\item 
andere ?
\end{itemize}
