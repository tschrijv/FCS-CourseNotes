\chapter{Voorwoord}


Algoritmen staan centraal in de computerwetenschappen. Effectieve
algoritmen voor concrete problemen steunen altijd op een goede keuze
van de abstractie en formeel inzicht in die abstractie. Het blijkt dat
{\bf grafen} dikwijls van pas komen en daarom begint deze cursus met
een stukje grafentheorie: basisbegrippen i.v.m. grafen, een paar
stellingen, en wat algoritmen die steunen op stellingen. In de
daaropvolgende hoofdstukken komen die regelmatig van pas. Na
grafentheorie volgen twee hoofdstukken over {\bf talen} en {\bf
beslissingsproblemen}: een hi\"erarchie van talen wordt opgebouwd aan
de hand van de machinerie die nodig is om een taal te beslissen en het
formalisme om een taal te specifi\"eren. Bovendien wordt ruim aandacht
besteed aan de niet-beslisbare talen: dit geeft de limieten aan van
wat mogelijk is met een algoritme. De cursus eindigt met een inleiding
tot complexiteitstheorie: daarin wordt \'{e}\'{e}n van de centrale
problemen van de computerwetenschappen formeel aangebracht, namelijk
de vraag of $\P$ gelijk is aan $\NP$. Behalve tijdscomplexiteit komt
ook ruimtecomplexiteit aan bod. 


Deze tekst bevat stukken uit de cursustekst voor de vakken
\begin{verse}
* {\em Fundamenten voor de Informatica} 1$^{ste}$ Bachelor
Informatica, geschreven samen met collega K. De Kimpe in 1997,

* {\em Automaten en Berekenbaarheid} 3$^{de}$ Bachelor Informatica, geschreven in 2004.
\end{verse}
Dat verklaart de niet-uniforme stijl.

Geraadpleegde bronnen en bijkomend materiaal staan soms op het einde
van een hoofdstuk, soms op het einde van een onderwerp, soms in de
tekst zelf. Oefeningen zijn met opzet niet opgenomen, maar er staan
wel regelmatig {\em selfies} in: dingen om zelf te doen.
